\section{Instance Definition}

In this section, we introduce a traffic light located in southern Odense, Denmark and its implementation in the microscopic traffic simulator SUMO - \textit{Simulation of Urban MObility}. 

Overview: 
Introduction to topology (as given by \swarco)
Introduction to SUMO
SWARCO implemented in SUMO
SWARCO data & data processing

\subsection{Network topology}
The company \swarco provided the data and network information we use in this project. We look at a traffic light controlled intersection located in southern Odense, Denmark, depicted in \replace{reference}. We modeled the intersection in the microscopic traffic simulator SUMO, without the inclusion of bicycles, as the available data on bicycles is quite limited.
As the intersection is within city bounds, a maximum speed of 50 km/h is used. \replace{check the actual speed limit at the intersections}

\begin{figure}[!htb]
  \replace{Anl427}
\end{figure}


\subsection{SUMO Introduction}
SUMO, short for \textit{Simulation of Urban MObility}, is a microscopic, well-established general-purpose traffic simulator. It has been around since 2001 and is an open source project. What makes it particularly interesting for this project, is the ability to control elements of the simulation externally, through a well-defined API. With this API, gathering information from the simulation for our agent is made simple, while also providing a way for our agent to control each traffic light.

\subsection{Implementing the intersection in SUMO}
SUMO uses a directed graph representation to define a traffic network, with some extra information. 
\textit{Nodes}  in the graph are points connected by one or more \textit{edges}. Nodes contain connection data, which is (potentially) a many-many mapping of incoming lanes to outgoing lanes. The edges carry the lane information, allowing any number of adjacent lanes (within computational reason) to follow any single edge between two nodes. As such, each edge defines a set of up- or down-stream lanes, possibly consisting of multiple traffic movements.


\replace{picture of SUMO network implementation}




In this section, we will introduce a traffic network consisting of three traffic light controlled
junctions, and its implementation in SUMO, an urban traffic simulator with a
powerful Traffic Control Interface ``TraCI''.
The junctions reside in southern Odense, Denmark.
As all junctions are within city bounds, the top speed for all lanes is 50 km/h.
A graphical representation of the junctions is available in \replace{reference}.



\subsection{SUMO Introduction}
SUMO, short for \textit{Simulation of Urban MObility}, is a well-established
general-purpose traffic simulator. It has been around since 2001, and is an open
source project. What makes it particularly interesting for this project, is the
ability to control elements of a simulation remotely, through a well-defined
API. This allow us go gather information from the simulation for our
\replace{agent(s)}, while also providing a way for our \replace{agent(s)} to control
each trafic light.

SUMO uses a directed graph representation to define a traffic network, with some extra
information. \textit{Nodes}  in the graph are points connected by one or more
\textit{edges}, which contain information on each lane
following the edge. Additionally, at each node, a \textit{connection}
describes how a lane on an incoming edge relates to a lane on an outgoing edge.

\subsection{Abstraction for use in SUMO}
Before we can implement the network in SUMO, we must abstract the features of the network into a graph representation consistent with that which SUMO supports.
We can describe this process through a number of steps:
\begin{enumerate}
\item Node definition
\item Edge and lane definition
\item Edge connection definition
\item Induction loop / detector definition
\item Traffic light phase definition
\end{enumerate}

\replace{Some more text?}

\subsubsection{Node Definition}
\replace{Mostly no longer relevant}

As mentioned, we will need a \tnode for each of the three junctions, but also a
few others.
In SUMO, roads are defined by edges between \textit{nodes}. Each edge has some
number of lanes, which is constant, hence a \tnode must be place at any point at
which the number of lanes or conditions on existing lanes change.
Additionally, many traffic light regulated junctions which include bike only
lanes require the bikes to go to an intermediate point subject to another
traffic light. Such intermediate points must also be included as ``\textit{dummy}''
nodes.

In this project node identifiers are chosen based on the relevant movement(s)
their edges represent, as well as a description of \textit{why} the node is
there. These are not requirements or standards set by SUMO, but a way to make
naming of the nodes more consistent. The naming scheme is as follows:

\begin{verbatim}
(closest junction id)(TL group id)(objective)_<index>
\end{verbatim}

\begin{table}[!htb]
  \centering
  \begin{tabular}{|l|l|}\hline
    ID & Description \\\hline
    Anl401 & Odensevej-Stenløsevej, Southernmost\\
    Anl427 & Odensevej-Landbrugsvej, Inbetween\\
    Anl411 & Hjallesevej-Niels Bohrs Allé, Northernmost\\\hline
  \end{tabular}
  \caption{Junction identifiers}\label{tab:jid}
\end{table}

The identifiers for the three junctions are as provided by the company
\textsc{swarco}, and can be seen in \cref{tab:jid}. In the sketches provided, traffic lights are grouped. We
include one of these group identifiers in the node identifier. The TL group
identifiers consist of a captial letter followed by a singe digit, and possibly
the following information: \texttt{V} for Left, \texttt{H} for right,
\texttt{Cy} for bike. Combinations are also possible e.g. \texttt{A1CyH}
indicates a right-turn bikelane from \texttt{A1}.\\


The \texttt{objective} part of the naming scheme consist of two things.\\
A modifier, which can be \texttt{a} for addition of a lane, \texttt{c} for
changing conditions on a lane, \texttt{r} for removal of a lane, \texttt{o} when
a node is needed for some other reason than changing lanes, usually shaping the streets, and \texttt{s} for
splitting a single lane into right- and left-turn lanes.\\
A lane code, which describes the function of the relevant lane. The
lane codes used are a subset of codes made available by the danish road
municipality ``Vejdirektoratet''. For our purposes, only a subset of these codes
are needed, as seen in \cref{tab:cse}.

\begin{table}[!htb]
  \centering
  \begin{tabular}{|l|l|}\hline
    Code & Description\\\hline
    4 & Lane\\ 
    5 & Right lane\\
    9 & Left lane\\
    10 & Right-turn lane 1\\
    11 & Right-turn lane 2\\
    13 & Left-turn lane 1\\
    16 & Others\\\hline
  \end{tabular}
  \caption{Selected Cross section elements\replace{reference}}
  \label{tab:cse}
\end{table}



\replace{Example with text for each node}


\subsubsection{Edge and lane Definition}
\replace{Arguably no longer relevant either}

In SUMO, road segments are represented by edges. 
The only mandatory field in the final network file, is the \texttt{id} field. 
However, when we construct the network, we use intermediate files to keep nodes, edges, connections and additionals separate. 
As such, we also have to keep track of the to and from
 fields, indicating which nodes the edge connects. Additionaly, edges contain information on lanes, such as number of lanes and width of each lane.\newline

We can arbitrarily give edges id's, as long as they are unique. The scheme we
use is ``(\texttt{junction id)-(locally unique id)}''. We can do this, as all the
information we need is stored in the \texttt{from} and \texttt{to} fields.

\subsection{Input data}
\replace{Remove the "overcounts, no misses part?}

Along with technical drawings of the intersection, files describing the traffic flow was provided in the form of 5 minute aggregated readings from the 25 induction loops present in the intersection.
To use this data in a meaningful way, a way of translating sets of detectors into routes a vehicle can follow has been made. 

Populating these sets with the input data, can be modeled as an integer linear programming problem, and draws many parallels with the generalized set covering problem.
Unfortunately, the induction loop readings has proven non-perfect, so we propose two different models.

The first model assumes that the induction loops never miss any vehicles, but may overcount.
The second model assumes that the induction loops don't overcount, but may miss vehicles. 
In reality, a mixture of the solutions from the two models probably comes closest to the actual number of vehicles. 

Both models give a vehicle count for each route for a single 5-minute interval, and should be called for each 5-minute interval.
This gives the advantage that if desired, under/over-counts can be carried over between solves, or ignored -- all while keeping the vehicle arrival rate granular enough to properly model rush hours in the final simulator instance.

\subsubsection{Overcounts, no misses}
Given a set of routes $R$, a set of detectors $D$, a binary route representation as defined below by $B_{ij}$, and upper bounds for \textit{total} number of vehicles passing a detector, defined below by $C_i$. 
Select the number of vehicles to follow each route, such that the number of vehicles passing each detector is maximized for all detectors, given the single constraint:
\begin{itemize}
  \item No more than $C_i$ vehicles can pass detector $D_i$, $\forall i \in \set{1, 2, \ldots, |D|}$
\end{itemize}

For convenience we define the sets $I = \set{1, 2, \ldots, |D|}$, and $J = \set{1, 2, \ldots, |R|}$.

Let $B{ij}$ be a binary constant such that:
\[
  B_{ij} = \begin{cases}
    1 & \text{if route j passes detector i}\\
    0 & \text{otherwise}
  \end{cases}\quad \forall i \in I, j \in J
\]
Let $C_i$ be an an detected number of vehicles at detector $D_i,\ \forall i \in I$
Let $x_j$ be an integer variable with a lower bound of 0, indicating the number of vehicles following route $j, \forall j \ in J$
\begin{align*}
  \begin{array}{rrcll}
    \text{Maximize} & \sum_{j\in J}x_j \cdot \sum_{i\in I}B_{ij}&&&\\
    \text{s.t.} & \sum_{j\in J}B_{ij}\cdot x_j & \leq & C_i & \qquad \forall i \in I\\
    & x_j & \in & \mathbb{N}&\qquad \forall j \in J\\
    & x_j & \geq & 0&\qquad \forall j \in J
  \end{array}
\end{align*}
The objective function maximizes the total sum of vehicles to insert, by multiplying the number of vehicles on route some route, by the sum of detectors it passes. 

The first constraint ensures that the sum of vehicles passing a detector does not surpass its capacity.

The second constraint ensures that only an integer number of vehicles can be added to the system.

The Third constraint ensures that the number of cars on each route must be non-negative.

\subsubsection{Misses, no overcounts}